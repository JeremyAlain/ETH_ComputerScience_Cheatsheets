\subsection{Jacobi Matrix}
Die Jacobimatrix (oder Funktionalmatrix) einer Funktion $F$ besteht aus den ersten partiellen Ableitungen von allen Komponenten nach allen Variabeln. Also aus den Gradienten von jeder Komponente. Sei $F: \R^n \to \R^m$, dann gilt:
\begin{align*}
	JF = DF =
	\left(\begin{array}{cccc}
		\vspace{0.2cm}\frac{\partial F_1}{\partial x_1} & \frac{\partial F_1}{\partial x_2} & \ldots & \frac{\partial F_1}{\partial x_n}\\
		\vspace{0.1cm}\frac{\partial F_2}{\partial x_1} & \frac{\partial F_2}{\partial x_2} & \ldots & \frac{\partial F_2}{\partial x_n}\\
		\vspace{0.05cm}\vdots & \vdots & & \vdots\\
		\frac{\partial F_m}{\partial x_1} & \frac{\partial F_m}{\partial x_2} & \ldots & \frac{\partial F_m}{\partial x_n}
	\end{array}\right)
	=
	\left(\begin{array}{c}
		\vspace{0.2cm} - \nabla F_1 -\\
		\vspace{0.1cm} - \nabla F_2 -\\
		\vspace{0.05cm}\vdots\\
		- \nabla F_m -
	\end{array}\right)
\end{align*}

\subsubsection{Jacobi Determinante}
Ist einfach die Determinante der Jacobimatrix. Will man das Integral über einen Bereich ausrechnen, welcher in einem anderen Koordinatensystem einfacher ist, dann muss zusätzlich zur Koordinatentransformation, das Integral mit der Jacobideterminante für dieses neue Koordinatensystem multipliziert werden!\\
Bsp. Für die Polarkoordinaten ist die Jacobi Determinante:
\[
	F = 
	\left(\begin{array}{c}
		r \cdot \cos(\phi) \\
		r \cdot \sin(\phi)
	\end{array}\right)
	\Rightarrow
	|JF| = 
	\left|\begin{array}{cc}
		\cos(\phi) & -r \cdot \sin(\phi)\\
		\sin(\phi) & r \cdot \cos(\phi)\\
	\end{array}\right|
	= r
\]
\section{Green's Theorem}
Beschreibt das Verhältnis von Rotor/Rotation(Mikroskopische/Makroskopisch). Das Wegintegral kann über das Zusammenaddieren von mikroskopischen Rotationen(Rotor) um alle einzelnen Punkte innerhalb des Gebietes berechnet werden.\\
$\int_{C}^{} = \int \int_{D}^{} rot(F) \cdot k dA$ \textbf{$= \int \int_{D}^{} \frac{\partial F_2}{\partial x} - \frac{\partial F_1}{\partial y}$} \\ wobei k der Normale Vektor in z-Richtung ist.\\ $\newline$
Bsp: Berechne gegebene Integrale der 1-Form als Wegintegral und dann mit Green:
$\int_{\partial A}^{} (x^2 dx + y^2dy),$ \\A $= [{(x,y) \in R^2; -\frac{\pi}{2} \leq x \leq \frac{\pi}{2}, -1 \leq y \leq cos(x)}]$\\

\textbf{Wegintegral:}$\int_{\partial A}^{} \lambda = \int_{\lambda _1}^{} \lambda + \int_{\lambda _2}^{} \lambda + \int_{\lambda _3}^{} \lambda + \int_{\lambda _4}^{} \lambda =$...$\\$(siehe Wegintegrale)... $=\left.[\frac{1}{3} cos^3(t)] \right|_{-\frac{\pi}{2}}^{\frac{\pi}{2}} = 0$\\
\textbf{Satz von Green:} $\int_{\partial A}^{} \lambda = \int_{\partial A}^{} (F_1(x,y)dx + F_2(x,y)dy) =\\ \int_{A}^{}(\frac{\partial F_2}{\partial x} - \frac{\partial F_1}{\partial y}) d\mu = 0$

\section{Stoke's Theorem} Erweiterung von Green's Theorem über eine Kurve auf eine Fläche(also $R^2 \rightarrow R^3$).\\
$\int_{C}^{} F \cdot ds = \int \int_{S}^{} curl(F) \cdot n dS = \int \int_{S}^{} \left(\begin{array}{c}
\vspace{3pt} F_{3_y} - F_{2_z} \\
\vspace{3pt} F_{1_z} - F_{3_x} \\
F_{2_x} - F_{1_y}
\end{array}\right)$$\cdot n \cdot dS$

Algorithmus: \textbf{1.} Berechne den Curl(F)\\
\textbf{2.} Berechne den Normalenvektor auf die Oberfläche\\
\textbf{3.} Berechne das Integral $\int_{0}^{D}$ wobei D die Oberfläche ist. 


\section{Koordinatensysteme}
\vspace{-0.1cm}\subsection{Koordinaten im $\R^2$}
\textbf{Polarkoordinaten $(r, \phi)$:}\\
\begin{align*}
	x &= r \cdot \cos(\phi) &  r &= \sqrt{x^2 + y^2} & 0 &\leq r \leq \infty\\
	y &= r \cdot \sin(\phi) &  \phi &= arg(x, y) & 0 &\leq \phi < 2\pi
\end{align*}

\begin{minipage}{0.6\columnwidth}
	Jacobideterminante: $r$\\
	Integralsubstitution: $dx \, dy \to r \, dr \, d\phi$
\end{minipage}
\begin{minipage}{0.39\columnwidth}
	\includegraphics[width=\columnwidth]{polar_coordinate}
\end{minipage}

\vspace{-0.3cm}\subsection{Koordinaten im $\R^3$}
\textbf{Kugelkoordinaten I $(r, \vartheta, \phi)$:}\\
\begin{align*}
	x &= r \cdot \sin(\vartheta) \cdot \cos(\phi) & r &= \sqrt{x^2 + y^2 + z^2} & 0 &\leq r \leq \infty\\
	y &= r \cdot \sin(\vartheta) \cdot \sin(\phi) & \phi &= arg(x, y) & 0 &\leq \phi < 2\pi\\
	z &= r \cdot \cos(\vartheta) & \vartheta &= \arccos({\small \frac{z}{\sqrt{x^2 + y^2 + z^2}}}) & 0 &\leq \vartheta \leq \pi
\end{align*}
Jacobideterminante: $r^2 \cdot \sin(\vartheta)$\\
Integralsubstitution: $dx \, dy\, dz \to r^2 \cdot \sin(\vartheta) \, dr \, d\phi \, d\vartheta$

\textbf{Kugelkoordinaten II $(r, \vartheta, \phi)$:}\\
\begin{align*}
	x &= r \cdot \cos(\vartheta) \cdot \cos(\phi) & r &= \sqrt{x^2 + y^2 + z^2} & 0 &\leq r \leq \infty\\
	y &= r \cdot \cos(\vartheta) \cdot \sin(\phi) & \phi &= arg(x, y) & 0 &\leq \phi < 2\pi\\
	z &= r \cdot \sin(\vartheta) & \vartheta &= \arcsin({\small \frac{z}{\sqrt{x^2 + y^2 + z^2}}}) &\hspace{-0.15cm}-\frac{\pi}{2} &\leq \vartheta \leq \frac{\pi}{2}
\end{align*}
Jacobideterminante = $r^2 \cdot \cos(\vartheta)$\\
Integralsubstitution: $dx \, dy\, dz \to r^2 \cdot \cos(\vartheta) \, dr \, d\phi \, d\vartheta$

\begin{minipage}{0.5\columnwidth}
	\includegraphics[width=\columnwidth]{kugel_coordinates_I.png}\\
	Kugelkoordinaten I
\end{minipage}
\begin{minipage}{0.5\columnwidth}
	\includegraphics[width=\columnwidth]{kugel_coordinates_II.png}\\
	Kugelkoordinaten II
\end{minipage}

\textbf{Zylinderkoordinaten $(r, \phi, z)$:}\\
\begin{align*}
	x &= r \cdot \cos(\phi) &  r &= \sqrt{x^2 + y^2} & 0 &\leq r \leq \infty\\
	y &= r \cdot \sin(\phi) &  \phi &= arg(x, y) & 0 &\leq \phi < 2\pi\\
	z &= z & z &= z & -\infty & \leq z \leq \infty
\end{align*}
\begin{minipage}{0.68\columnwidth}
	Jacobideterminante: $r$\\
	Integralsubstitution: $dx \, dy \, dz \to r \, dr \, d\phi \, dz$
\end{minipage}
\begin{minipage}{0.31\columnwidth}
	\hspace{-0.3cm}\includegraphics[width=1.1\columnwidth]{cylindrical_coordinates}
\end{minipage}

\textbf{arg(x, y) für $0 \leq \phi < 2\pi $}\\
$arg(x, y) =
\begin{cases}
	\arctan(\frac{y}{x})				& x > 0 \gtext{und} y \geq 0 \\
	\arctan(\frac{y}{x}) + 2\pi			& x > 0 \gtext{und} y < 0 \\
	\arctan(\frac{y}{x}) + \pi			& x < 0\\
	\frac{\pi}{2}						& x = 0 \gtext{und} y > 0\\
	\frac{3\pi}{2}						& x = 0 \gtext{und} y < 0\\
	0 									& x = 0 \gtext{und} y = 0
\end{cases}
$

\textbf{arg(x, y) für $-\pi < \phi \leq \pi$}\\
$arg(x, y) =
\begin{cases}
	\arctan(\frac{y}{x})				& x > 0 \\
	\arctan(\frac{y}{x}) + \pi			& x < 0 \gtext{und} y \geq 0 \\
	\arctan(\frac{y}{x}) - \pi			& x < 0 \gtext{und} y < 0 \\
	\frac{\pi}{2}						& x = 0 \gtext{und} y > 0\\
	-\frac{\pi}{2}						& x = 0 \gtext{und} y < 0\\
	0 									& x = 0 \gtext{und} y = 0
\end{cases}
$\\
\begin{minipage}{0.68\columnwidth}
	\textbf{Beispiel:}\\
	Berechne die Fläche des Halbkreisrings\\
	$R: 1 \leq x^2 + y^2 \leq 4$ und $y \geq 0$:
\end{minipage}
\begin{minipage}{0.31\columnwidth}
	\includegraphics[width=\columnwidth]{example_polar_coordinates}
\end{minipage}

Die Fläche lässt sich mit Polarkoordinaten besonders einfach ausrechnen. Die Parametrisierung für $r$ und $\phi$ lässt sich leicht aus der Grafik ablesen:
\begin{align*}
	x &= r \, \cos(\phi)	&& 1 \leq r \leq 2 \\
	y &= r \, \sin(\phi)	&& 0 \leq \phi \leq \pi
\end{align*}
Das Flächenintegral ist somit:
\begin{align*}
	R = \int_R dx \, dy = \int_R r \, dr \, d\phi = \int_0^{\pi} \int_1^2 r \, dr \, d\phi = \pi \left[\frac{r^2}{2}\right]_1^2 = \uuline{\frac{3}{2} \pi}
\end{align*}

\subsection{Horner Schema}
$2x^3 + 4x^2 -2x -4$
\begin{enumerate}
	\item Nullstelle erraten $\rightarrow$ x = 1
	\item Berechnung
	\begin{align*}
	 &&x^3 	&& x^2 && x^1 && x^0 \\
	 &&2 && 4 && -2 && -4 \\
	for\,x= 1 &&  && 2 && 6 && 4
	\end{align*}
	\begin{enumerate}
		\item Berechnung: $ 2 + 1 \cdot 4 = 6$
		\item Berechnung $ 6 + 1 \cdot -2 = 4$
		\item Berechnung $ 4 + 1\cdot -4 = 0$
	\end{enumerate}
\end{enumerate}
$\rightarrow 2x^2 + 6x + 4$

\subsection{Divergenzsatz im $\R^2$}
\begin{minipage}{0.68\columnwidth}
	Sei $G \subset \R^2$ eine Fläche mit Randkurve $C$ und äusserem Normalenvektor $\vec{n}$. $F$ sei ein Vektorfeld. Dann gilt:
	\[
		\int_{C} F \cdot \vec{n} \, ds = \int\int_G \operatorname{div} F \, dx \, dy
	\]
\end{minipage}
\begin{minipage}{0.31\columnwidth}
	\includegraphics[width=\columnwidth]{divergenzsatz_r2}
\end{minipage}
Dh: Der Fluss des Vektorfelds durch die Randkurve $C$, ist gleich dem Integral der Quellenstärke im Innern der Fläche $G$!\\

\begin{minipage}{0.75\columnwidth}
	\textbf{Beispiel:}\\
	Berechne den Fluss von 
	$F(x, y) = 
	\left(
	\begin{array}{c}
		x^3\\
		0 
	\end{array}\right)$
	durch den Rand des Kreises mit Radius 2:
\end{minipage}
\begin{minipage}{0.24\columnwidth}
	\includegraphics[width=\columnwidth]{example_divergence_theorem}
\end{minipage}
Dazu verwenden wir den Divergenzsatz (Satz von Gauss) und nehmen Polarkoordinaten:
\begin{align*}
	\operatorname{div} F &= 3x^2 && 0 \leq r \leq 2 \gtext{und} 0 \leq \phi \leq 2\pi \\
	\int_{C} F \cdot \vec{n} \, ds &= 
	\int\int_G 3x^2 \, dx \, dy &&= 
	\int_0^{2\pi} \int_0^2 3 r^2 \cos(\phi)^2 \, r \; dr \, d\phi \\
	& &&= 3 \left[ \frac{r^4}{4} \right]_0^2 \cdot \left[ \frac{\phi}{2} + \frac{1}{4} \sin(2\phi) \right]_0^{2\pi} \\
	& &&= 3 \cdot 4 \cdot \pi = 12 \pi
\end{align*}
 
\subsection{Divergenzsatz im $\R^3$}
\begin{minipage}{0.68\columnwidth}
	Sei V $\subset \R^3$ ein Gebiet mit Randfläche (Oberfläche) S = $\partial V$ und äusserem Flächennormalenvektor $\vec{n}$. $F$ sei ein Vektorfeld. Dann ist:
	\[
		\int \int_S F \cdot \vec{n} \; dS = \int \int \int_V \operatorname{div} F \; dx \, dy \, dz
	\]	
\end{minipage}
\begin{minipage}{0.31\columnwidth}
	\includegraphics[width=\columnwidth]{divergence_theorem_r3}
\end{minipage}
Dh: Der Fluss des Vektorfelds durch die Randfläche des Gebiets ist gleich dem Integral der Quellenstärke im Innern!

\section{Weitere Beispiele}
\begin{footnotesize}
\textbf{Bsp. Zwischenwertsatz}\\
Seien $c, d \in \R$ und $c < d$. Zeige mit Zwischenwertsatz, dass die nachfolgende Gleichung eine Lösung im Intervall $]c, d[$ hat.
\[
	\frac{2}{(x - c)^4} + \frac{5}{(x - d)^9} = 0
\]
Wir definieren $f(x) = \frac{2}{(x - c)^4} + \frac{5}{(x - d)^9}$\\
Gesucht ist also ein $x \in ]c, d[$, so dass die Gleichung erfüllbar ist! 
Dazu suchen wir zwei Punkte $x_1, x_2$ mit $c < x_1 < x_2 < d$ so dass $f(x_2) < 0 < f(x_1)$   (oder $f(x_1) < 0 < f(x_2)$).
\[
	\lim_{x \to c^{+}} \ucomment{+ \infty}{\frac{2}{(x - c)^4}} + \ucomment{const.}{\frac{5}{(x - d)^9}} = + \infty
\]
$\Rightarrow$ Wir finden also sicher ein $x_1$ mit $c < x_1 < \frac{c + d}{2}$ und $f(x) > 0$!
\[
	\lim_{x \to d^{-}} \ucomment{const.}{\frac{2}{(x - c)^4}} + \ucomment{- \infty}{\frac{5}{(x - d)^9}} = - \infty
\]
$\Rightarrow$ Wir finden also sicher ein $x_x$ mit $\frac{c + d}{2} < x_2 < d$ und $f(x) < 0$!

Da die Funktion stetig ist, und wir zwei Punkte $x_1 < x_2$ gefunden haben für die gilt $f(x_2) < 0 f(x_1)$ folgt aus dem Zwischenwertsatz, dass es ein $x \in ]c, d[$ geben muss sodass $f(x) = 0$!
\end{footnotesize}
\subsection{Konvexität}
Eine Funktion ist Konvex wenn gilt $f(\frac{a+b}{2}) \leq \frac{f(a) + f(b)}{2}$
Oder auch
$f(tx +(1-t)y) \leq tf(x) + (1-t) f(y)$
Die Konvexe Funktion hat unten ein Minimum und die Konkave oben ein Maximum. 

