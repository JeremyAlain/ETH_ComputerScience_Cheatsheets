% ------------------------------------------------------------------------------------------------ %
% GRUNDLAGEN
% ------------------------------------------------------------------------------------------------ %


\section{Grundlagen}

\begin{definition}[Stichprobe]
Die Gesamtheit der Beobachtungen $x_1, \ldots, x_n$ oder der Zufallsvariablen $X_1, \ldots, X_n$ wird \emph{Stichprobe} genannt; die Anzahl $n$ heisst \emph{Stichprobenumfang}.
\end{definition}

\begin{definition}[Empirische Verteilungsfunktion]
Die \emph{empirische Verteilungsfunktion} $F_n$ zu den Messdaten $x_1,\ldots,x_n$ ist definiert durch
$$
F_n(y) := \frac{1}{n} \lvert\{x_i\mid x_i \leq y\}\rvert = \frac{1}{n} \sum_{i\text{ mit } x_i \leq y} f_i.
$$
\end{definition}

\begin{definition}[Empirischer Mittelwert]
$\newline$
Sch�tzer f�r $\mu \overline{x}_n = \overline{x} = \frac{1}{n} \sum_{i=1}^n x_i
$
\end{definition}

\begin{definition}[Empirische Varianz und Standardabweichung]
$\newline$
Sch�tzer f�r Varianz $s_n^2 = s^2 = \frac{1}{n-1} \sum_{i=1}^n (x_i-\overline{x})^2
$
\end{definition}

\begin{definition}[Empirisches Quantil] Das \emph{empirische $\alpha$-Quantil} zu den geordneten Daten $x_{(1)}, \ldots, x_{(n)}$ ist gegeben durch
$$
(1-\alpha) x_{(k)} + \alpha x_{(k+1)} = x_{(k)} + \alpha \left(x_{(k+1)} - x_{(k)} \right),
$$
wobei $k = \lfloor \alpha n \rfloor$ und $\alpha \in (0,1)$.
Damit liegt etwa der Anteil $\alpha$ unterhalb des empirischen $\alpha$-Quantils, und somit etwa der Anteil $1-\alpha$ oberhalb.
\end{definition}

\begin{definition}[Empirischer Median]
Der \emph{empirische Median} ist definiert als das $0.5$-Quantil.
\end{definition}
