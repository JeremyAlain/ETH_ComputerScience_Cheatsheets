% ------------------------------------------------------------------------------------------------ %
% Erwartungswert
% ------------------------------------------------------------------------------------------------ %


\subsection{Erwartungswert}

\begin{definition}[Diskreter Erwartungswert]
Ist $X$ diskrete Zufallsvariable mit Gewichtsfunktion $p_X$, so ist der \emph{Erwartungswert} von $X$ definiert als
$$
\E[X] := \sum_{x_i \in \mathcal{W}(X)} x_i p_X(x_i),
$$
sofern diese Reihe konvergiert.
$$
\E[X+Y] = \E[X] + \E[Y]
$$
$$
\E[XY] = E[X]*E[Y]
$$
Sofern die Zv unabh�ngig sind.
Beachte: Bei Varianz gilt dies nur falls Unabh�ngigkeit vorhanden ist. 
\end{definition}
Beachte: Falls die Gewichtsfunktion aufsummiert $!= 1 \Longrightarrow$ keinen Erwartungswert.$\newline$
Falls man mehrere Zufallsvariablen hat, benutzt man f�r den Erwartungsert je nach dem die Randdichte. 

\begin{definition}[Stetiger Erwartungswert]
Falls $X$ eine stetige Zufallsvariable mit Dichte $f_X$ ist, dann ist der \emph{Erwartungswert} von $X$ definiert als
$$
\E[X] := \int_{-\infty}^\infty x f_X(x) \d x,
$$
falls das Integral konvergiert.
\end{definition}

\begin{theorem}[4.1]
Sei $X$ eine diskrete Zufallsvariable mit Gewichtsfunktion $p_X$ und $Y := g(X)$, dann gilt
$$
\E[Y] = \E[g(X)] = \sum_{x_i \in \mathcal{W}(X)} g(x_i) p_{X}(x_i).
$$
Falls $X$ eine stetige Zufallsvariable mit Dichte $f_X$ ist, ist analog
$$
\E[Y] = \E[g(X)] = \int_{-\infty}^\infty g(x) f_X(x) \d x
$$
\end{theorem}
\begin{example}[Beispiel Diamant]
$\E[XY] = \int \int_{Q}^{} xyf_{X,Y}(x,y)dxdy \\
=\int_{-\infty}^{\infty}\int_{-\infty}^{\infty}xy\frac{1}{2}1_{\{(x,y) \epsilon Q\}}dxdy \\
= \frac{1}{2} \int_{-\infty}^{\infty} \int_{-\infty}^{\infty} xy\cdot 1_{\{-1\leq x\leq 1 und -(1-\abs{x})\leq y\leq(1-\abs{x})\}}dxdy\\
= \frac{1}{2} \int_{-\infty}^{\infty} \int_{-\infty}^{\infty} xy\cdot 1_{\{-1\leq x\leq 1\}}1_{\{-(1-\abs{x})\leq y\leq(1-\abs{x})\}}dxdy\\
= \frac{1}{2} \int_{-\infty}^{\infty}x1_{\{-1\leq x \leq 1\}}dx(\int_{-\infty}^{\infty}y1_{\{-(1-\abs{x})\leq < \leq(1-\abs{x})\}}dy)\\
= \frac{1}{2} \int_{-1}^{1}xdx\underbrace{(\int_{-(1-\abs{x})}^{1-\abs{x}}ydy)}_{=0}$

\end{example}


% ------------------------------------------------------------------------------------------------ %
% VARIANZ UND STANDARDABWEICHUNG
% ------------------------------------------------------------------------------------------------ %

\subsection{Varianz und Standardabweichung}

\begin{definition}[Varianz]
Sei $X$ eine Zufallsvariable mit $\E[X^2] < \infty$. Die \emph{Varianz} von $X$ ist definiert als
$$
\var[X] := \E[X^2] - (\E[X])^2
$$
$$
\var[X] := \E\left[(X-\E(X))^2\right].
$$
$$
\var[Y] = \var[aX+b] = a^2\var[X]
$$
\end{definition}

\begin{note} Nach Satz 4.1 l�sst sich die Varianz folgendermassen berechnen:
$$
\var[X] = \int_{-\infty}^\infty (x-\E[X])^2 f_X(x) \d x
$$
\end{note}

\begin{definition}[Standardabweichung]
Die \emph{Standardabweichung} einer Zufallsvariable $X$ ist
$$
\sigma_X := \sqrt{\var[X]}.
$$
\end{definition}


% ------------------------------------------------------------------------------------------------ %