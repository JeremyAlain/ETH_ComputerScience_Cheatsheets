% ------------------------------------------------------------------------------------------------ %
% ANHANG
% ------------------------------------------------------------------------------------------------ %

\part*{Anhang}
\setcounter{part}{1}
\setcounter{section}{0}
\addcontentsline{toc}{part}{Anhang}
\iffalse
\section{Reihen und Integrale}
Bei den folgenden Integralen wurden die Integrationskonstanten weggelassen.
$$
\begin{array}{rcll}
\int a                  \,\d x & = & ax                                                           &\\[0.5em]
\int x^a                \,\d x & = & \frac{1}{a+1} x^{a+1}                                        ,&a \neq -1 \\[0.5em]
\int (ax+b)^c           \,\d x & = & \frac{1}{a(c+1)}(ax+b)^{c+1}                                 ,&c \neq -1 \\[0.5em]
\int \frac{1}{x}        \,\d x & = & \log\abs{x}                                                  ,&x\neq 0\\[0.5em]
\int \frac{1}{ax+b}     \,\d x & = & \frac{1}{a} \log\abs{ax+b}                                   &\\[0.5em]
\int \frac{1}{x^2+a^2}  \,\d x & = & \frac{1}{a}\arctan\frac{x}{a}                                &\\[2em]
%
% EXPONENTIAL
%
\int e^{ax}             \,\d x & = & \frac{1}{a}e^{ax}                                            &\\[0.5em]
\int x e^{ax}           \,\d x & = & \frac{e^{ax}}{a^2}(ax-1)                                     &\\[0.5em]
\int x^2 e^{ax}         \,\d x & = & e^{ax}\left(\frac{x^2}{a}-\frac{2x}{a^2}+\frac{2}{a^3}\right)&\\[2em]
%
% LOGARITHMUS
%
\int \log\abs{x}        \,\d x & = & x(\log\abs{x} - 1)                                           &\\[0.5em]
\int \log_a\abs{x}      \,\d x & = & x(\log_a\abs{x}-\log_a e)                                    &\\[0.5em]
\int x^a \log x         \,\d x & = & \frac{x^{a+1}}{a+1}\left(\log x - \frac{1}{a+1}\right)       ,&a\neq-1,x>0\\[0.5em]
\int \frac{1}{x}\log x  \,\d x & = & \frac{1}{2}\log^2 x                                          ,&x>0\\[2em]
%
% TRIGONOMETRISCHE FUNKTIONEN
%
\int \sin(ax+b)         \,\d x & = & -\frac{1}{a}\cos(ax+b)                                       &\\[0.5em]
\int \cos(ax+b)         \,\d x & = & \frac{1}{a}\sin(ax+b)                                        &\\[0.5em]
\int \tan x             \,\d x & = & -\log\abs{\cos x}                                            &\\[0.5em]
\int \frac{1}{\sin x}   \,\d x & = & \log\abs{\tan\frac{x}{2}}                                    &\\[0.5em]
\int \frac{1}{\cos x}   \,\d x & = & \log\abs{\tan(\frac{x}{2}+\frac{\pi}{4})}                    &\\[0.5em]
\int \sin^2 x           \,\d x & = & \frac{1}{2}(x-\sin x\cos x)                                  &\\[0.5em]
\int \cos^2 x           \,\d x & = & \frac{1}{2}(x+\sin x\cos x)                                  &\\[0.5em]
\int \tan^2 x           \,\d x & = & \tan x - x                                                   &\\[2em]
%
%
% ALLGEMEIN
%

\int \frac{f'(x)}{f(x)} \,\d x & = & \log\abs{f(x)}                                               &\\[2em]
\end{array}
$$

\subsection{Summen}
$$
\begin{array}{rcll}
\sum_{k=1}^n k                   & = & \frac{n(n+1)}{2}          &\\[0.5em]
\sum_{k=1}^n k^2                 & = & \frac{n(n+1)(2n+1)}{6}    &\\[0.5em]
\sum_{k=0}^n      a_0 q^k        & = & a_0 \frac{1-q^{n+1}}{1-q} &\\[0.5em]
\sum_{k=0}^\infty a_0 q^k        & = & \frac{a_0}{1-q}           &\\[0.5em]
\sum_{k=0}^\infty \frac{k}{a^k}  & = & \frac{a}{(a-1)^2}         ,&\abs{a} > 1\\[2em]
\sum_{k=0}^\infty \frac{x^k}{k!} & = & e^x                       &
\end{array}
$$
\fi



% ------------------------------------------------------------------------------------------------ %
% KOMBINATORIK
% ------------------------------------------------------------------------------------------------ %
\iffalse
\section{Kombinatorik}
Ziehen von $k$ Elementen aus einer Menge mit $n$ Elementen
\begin{center}
	\begin{tabular}{|c||c|c|}\hline
		& geordnet & ungeordnet \\\hline\hline
		mit zur�cklegen & $n^k$ & ${n+k-1 \choose k}$ \\\hline
		ohne zur�cklegen & $\frac{n!}{(n-k)!}$ & ${n \choose k}$ \\\hline
	\end{tabular}
\end{center}

\fi
% ------------------------------------------------------------------------------------------------ %
% REIHEN
% ------------------------------------------------------------------------------------------------ %


% ------------------------------------------------------------------------------------------------ %
% INTEGRALE
% ------------------------------------------------------------------------------------------------ %


\begin{definition}[Partielle Integration]
$
\int_a^b f'(x)g(x) \d x = \left[ f(x)g(x) \right]_a^b - \int_a^b f(x)g'(x) \d x
$
\end{definition}

\begin{definition}[Substitutionsregel]
$
\int_a^b f(g(x)) g'(x) \d x \overset{t=g(x)}{=} \int_{g(a)}^{g(b)} f(t) \d t
$
\end{definition}

% \begin{center}
% \begin{tabular}{|l|l|}\hline
% $f(x)$              & $F(x)$                            \\\hline\hline
% $\frac{1}{x^2+a^2}$ & $\frac{1}{a}\arctan\frac{x}{a}$   \\
% $x e^{cx}$          & $\frac{e^{cx}}{c^2}(cx-1)$        \\\hline

% TRIGONOMETRISCHE FUNKTIONEN

% $\sin(ax+b)$        & $-\frac{1}{a}\cos(ax+b)$          \\
% $\cos(ax+b)$        & $\frac{1}{a}\sin(ax+b)$           \\
% $\tan x$            & $-\log\lvert\cos x\rvert$         \\
% $\frac{1}{\sin x}$  & $\log\lvert\tan\frac{x}{2}\rvert$ \\
% $\frac{1}{\cos x}$  & $\log\lvert\tan(\frac{x}{2}+\frac{\pi}{4})\rvert$ \\
% $\sin^2 x$          & $\frac{1}{2}(x-\sin x\cos x)$     \\
% $\cos^2 x$          & $\frac{1}{2}(x+\sin x\cos x)$     \\
% $\tan^2 x$          & $\tan x - x$                      \\\hline
% \end{tabular}
% \end{center}




\textbf{Summenverschiebung:}\\
$\sum_{k=3}^{k=5} \frac{4k+2}{2} = \sum_{(k+2)=3}^{(k+2)=5}\frac{4(k+2)+2}{2} = \sum_{k=1}^{k=3}\frac{4k+8+2}{2}$\\
\textbf{Beispiel mit Summen}\\
$1 = \sum_{k=2}^{\infty}\sum_{j=1}^{k-1}\P[X=j,Y=k] \\
= C\cdot \sum_{k=2}^{\infty}\sum_{j=1}^{k-1}(\frac{1}{2})^k\\
= C\cdot \sum_{j=1}^{\infty}\sum_{k=j+1}^{\infty}(\frac{1}{2})^k\\
= C\cdot\sum_{j=1}^{\infty}(\frac{1}{2})^j+1 \sum_{k=0}^{\infty}(\frac{1}{2})^k \\
=C\cdot \sum_{j=1}^{\infty} \frac{(\frac{1}{2})^j+1}{1-\frac{1}{2}} = C\cdot \sum_{1}^{\infty}(\frac{1}{2})^j\\
= C\cdot \frac{\frac{1}{2}}{1-\frac{1}{2}}$

% ------------------------------------------------------------------------------------------------ %