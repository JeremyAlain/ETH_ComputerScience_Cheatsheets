% ------------------------------------------------------------------------------------------------ %
% FUNKTIONEN VON ZUFALLSVARIABLEN
% ------------------------------------------------------------------------------------------------ %


\subsection{Funktionen von Zufallsvariablen}

Ausgehend von den Zufallsvariablen $X_1,\ldots,X_n$ kann man mit einer Funktion $g:\R^n\rightarrow\R$ eine neue Zufallsvariable $Y = g(X_1,\ldots,X_n)$ bilden.

\begin{example}[Summe, diskret]
F�r die Gewichtsfunktion $p_Z$ der Summe $Z = X + Y$ zweier diskreten Zufallsvariablen $X$ und $Y$ mit gemeinsamer Gewichtsfunktion $p$ erh�lt man
$$
p_Z(z) =
\sum_{x_i \in \mathcal{W}(X)} \P[X=x_i,Y=z-x_i] = 
\sum_{x_i \in \mathcal{W}(X)} p(x_i,z-x_i)
$$
\end{example}

\begin{example}[Summe, stetig]
Sind $X$ und $Y$ stetige Zufallsvariablen mit gemeinsamer Dichte $f$, so ist die Verteilungsfunktion $F_Z$ der Summe $Z = X+Y$ gegeben durch
$$
F_Z =
\displaystyle\int_{-\infty}^\infty \int_{-\infty}^{z-x} f(x,y) \d y \d x \overset{v=x+y}{=}
\int_{-\infty}^z \int_{-\infty}^\infty f(x,v-x) \d x \d v
$$
und somit auch die Dichte
$$
f_Z = \frac{\d}{\d z} F_Z(z) = \int_{-\infty}^\infty f(x,z-x) \d x (=) $$ \\
$ \int_{-\infty}^{\infty} f_x(x)f_y(z-x)dx = \int_{0}^{z} \lambda e^{-\lambda x}\lambda e^{-\lambda(z-x)}dx = \int_{0}^{z} \lambda^2 e^{-\lambda z} dx = \lambda^2 ze^{-\lambda z}$
\end{example}

\begin{example}[Funktion von ZF's]
$\\ V := X+Y$ berechne nun $f_v\\$
Wir fixieren $v \in (0,\infty)$ und definieren $g(x,y):= 1_{x+y\leq v}\\$
$\P[V\leq v] = \E[g(X,Y)] = \lambda ^2 \int_{0}^{\infty}\int_{0}^{\infty} g(x,y)\cdot e^{-\lambda (x+y)}dydx\\$
$= \lambda ^2 \int_{0}^{\infty} e^{-\lambda x}(\int_{0}^{\infty} 1_{x+y \leq v}e^{-\lambda y}dy)dx\\$
$=\lambda ^2 \int_{0}^{\infty} e^{-\lambda x}1_{x\leq v}(\int_{0}^{\infty} 1_{x+y \leq v}e^{-\lambda y}dy)dx\\$
$=\lambda \int_{0}^{v} e^{-\lambda x}(\int_{0}^{v-x}\lambda e^{-\lambda y}dy)dx\\$
$=\lambda \int_{0}^{v} e^{-\lambda x}(1-e^{-\lambda(v-x)})dx\\$
$=1-e^{-\lambda v}-\lambda ve^{-\lambda v}$\\ Danach ableigen f�r Dichtefunktion.
\end{example}
\begin{example}[Min/Max]
$X_i$ sind unabh�ngige i.i.d verteile TF's mit absolut stetiger Dichtefunktion f und Verteilungsfunktion F.Bestimme $f_{(min)}$ von $X_{(min)}$\\
$F_{(n)}(t) = \P[X_{(n)}\leq t] = \P[X_1 \leq t,...,X_n \leq t ] = \prod_{k=1} \P[X_k \leq t] = (F(t))^n \rightarrow f_{(n)}(t) = \frac{d}{dt}F_{(n)}(t)=nF^{n-1}(t)f(t)$\\
F�r Minimum: $1-F_{(1)}(t)= 1-\P[X_{(1)}\ge t]= 1-\P[X_1 \ge t,...,X_n \ge t] = 1-(1-F(t))^n \rightarrow f_{(1)}(t) = \frac{d}{dt}F_{(1)}(t) = n(1-F(t))f(t)$

\end{example}
% ------------------------------------------------------------------------------------------------ %
% I.I.D.
% ------------------------------------------------------------------------------------------------ %


\subsubsection{i.i.d Annahme}

Die Abk�rzung \emph{i.i.d.} kommt vom Englischen \emph{independent and identically distributed}. Die $n$-fache Wiederholung eines Zufallsexperiments ist selbst wieder ein Zufallsexperiment. F�r die Zufallsvariablen $X_i$ der $i$-ten Wiederholung wird oft aus Gr�nden der Einfachheit Folgendes angenommen:
\begin{compactenum}[i:]
\item $X_1,\ldots,X_n$ sind paarweise unabh�ngig.
\item Alle $X_i$ haben dieselbe Verteilung.
\end{compactenum}


% ------------------------------------------------------------------------------------------------ %
% SPEZIELLE FUNKTIONEN VON ZUFALLSVARIABLEN
% ------------------------------------------------------------------------------------------------ %


\subsubsection{Spezielle Funktionen von Zufallsvariablen}

Wichtige Spezialf�lle sind die Summe $S_n = \sum_{i=1}^n X_i$ und das arithmetische Mittel $\overline{X}_n = \frac{S_n}{n}$.
\begin{compactenum}
\item Wenn $X_i \sim Be(p)$, dann ist $S_n \sim Bin(n,p)$.
\item Wenn $X_i \sim \mathcal{P}(\lambda)$, dann ist $S_n \sim \mathcal{P}(n\lambda)$.
\item Wenn $X_i \sim \mathcal{N}(\mu,\sigma^2)$, dann ist $S_n \sim \mathcal{N}(n\mu,n\sigma^2)$.
\end{compactenum}

F�r den Erwartungswert und die Varianz gilt allgemein
\begin{center}
\begin{tabular}{ll}
$\E[S_n] = n\E[X_i]$ &
$\var[S_n] = n\var[X_i]$
\end{tabular}
\end{center}


% ------------------------------------------------------------------------------------------------ %