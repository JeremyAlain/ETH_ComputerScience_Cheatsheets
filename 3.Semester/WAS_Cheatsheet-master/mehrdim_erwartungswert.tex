% ------------------------------------------------------------------------------------------------ %
% Erwartungswert
% ------------------------------------------------------------------------------------------------ %


\subsection{Erwartungswert}

\begin{note} Der Erwartungswert einer $n$-dimensionalen Verteilung wird als $n$-Tupel der Erwartungswerte aller Randverteilungen $\E[X_i]$ angegeben.
\end{note}

\begin{theorem}[4.2]
F�r den Erwartungswert $\E[Y]$ einer Funktion $Y := g(X_1, \ldots X_n)$ der Zufallsvariablen $X_1,\ldots,X_n$ gilt im diskreten Fall
$$
\E[Y] = \sum_{x_1,\ldots,x_n} g(x_1,\ldots,x_n) p(x_1,\ldots,x_n)
$$
und analog im stetigen Fall
$$
\E[Y] = \underset{\R^n}{\int\ldots\int} g(x_1,\ldots,x_n)f(x_1,\ldots,x_n) \d x_n \ldots \d x_1.
$$
\end{theorem}

\begin{theorem}[4.4]
Seien $X_1, \ldots, X_n$ Zufallsvariablen mit endlichen Erwartungswerten $\E[X_1],\ldots,\E[X_n]$, dann ist
$$
\E\left[a+\sum_{i=1}^n b_i X_i\right]
=
a + \sum_{i=1}^n b_i \E[X_i].
$$
\end{theorem}


% ------------------------------------------------------------------------------------------------ %
% KOVARIANZ UND KORRELATION
% ------------------------------------------------------------------------------------------------ %

\subsection{Kovarianz und Korrelation}

\begin{definition}[Kovarianz]
Seien $X$ und $Y$ Zufallsvariablen mit $\E[X^2] < \infty$ und $\E[Y^2] < \infty$, dann ist die \emph{Kovarianz} von $X$ und $Y$ gegeben durch
$$
\cov[X,Y] := \E[(X-\E(X))(Y-\E(Y))].
$$
\end{definition}

Es gelten folgende Rechenregeln:

\begin{compactenum}[i:]
\item $\var[X] = \E[X^2]-\E[X]^2$.
\item $\var[a+bX] = b^2 \var[X]$.
\item $\var[a+\sum_{i=1}^n b_iX_i] = \sum_{i=1}^n b_i^2 \var[X_i]$, f�r $X_i$ unabh�ngig.
\item $\var[X+Y] = \var[X]+\var[Y]+2\cov[X,Y]$.
\item $\cov[X,Y] = \cov[Y,X]$.
\item $\cov[X,X] = \var[X]$.
\item $\cov[X,Y] = \E[XY]-\E[X]\E[Y]$.
\item $\cov[X,a] = 0$ f�r alle $a \in \R$.
\item $\cov[X,bY] = b\cov[X,Y]$ f�r alle $b \in \R$.
\item $\cov[X,Y+Z] = \cov[X,Y] + \cov[X,Z]$.
\item $\cov[a+\sum_{i=1}^n b_iX_i,c+\sum_{j=1}^m d_jY_j]$ \\
      $=\sum_{i=1}^n\sum_{j=1}^m b_i d_j \cov[X_i,Y_j]$
\item $\cov[X,Y] = 0$, falls $X$ und $Y$ unabh�ngig.
\end{compactenum}


\begin{definition}[Korrelation]
Seien $X$ und $Y$ Zufallsvariablen, dann heisst
$$
\corr[X,Y] := \frac{Cov[X,Y]}{\sqrt{\var[X]}\sqrt{\var[Y]}}
$$
\emph{Korrelation} von $X$ und $Y$. Ist $\corr[X,Y] = 0$, oder �quivalent $\cov[X,Y]=0$, dann heissen $X$ und $Y$ unkorreliert.
\end{definition}

\begin{note}
Die Korrelation misst die St�rke und Richtung der \emph{linearen Abh�ngigkeit} zweier Zufallsvariablen $X$ und $Y$:
$$\corr[X,Y] = \pm 1 \,\Leftrightarrow\, \exists \, a \in \R, b>0: Y = a \pm bX$$
\end{note}

\begin{note}
Sind $X$ und $Y$ unabh�ngig, dann ist $\cov[X,Y]=0$ und $\corr[X,Y] = 0$. Die Umkehrung gilt aber im Allgemeinen nicht.
\end{note}


% ------------------------------------------------------------------------------------------------ %